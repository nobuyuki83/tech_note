\documentclass[12pt]{article}
\usepackage{geometry}
\geometry{a4paper}


\usepackage{color}
\usepackage{hyperref}
\usepackage{amsmath}
\usepackage{amsfonts}
\usepackage{amssymb}
\usepackage{graphicx}
\usepackage{tcolorbox}
\usepackage{listings}
\usepackage{here}
\usepackage{txfonts}
\usepackage{algorithm}
\usepackage{algorithmic}
\usepackage{siunitx}
\usepackage{xcolor}

\lstset {language = c++,
  basicstyle = \ttfamily \scriptsize,
  commentstyle = \textit,
  frame = tRBl,
  framesep = 5pt,
  showstringspaces = false,
  numbers = left,
  stepnumber = 1,
  numberstyle = \tiny,
  tabsize = 2,
  keywordstyle = \bfseries \color{blue},
  stringstyle=\color{magenta},
  commentstyle=\color{red},
  morecomment=[l][\color{red}]{\#}
  showstringspaces=false, % don't mark spaces in strings
}
\newcommand{\bi}[1]{\mathbf{#1}}
\newcommand{\bs}[1]{\boldsymbol{#1}}  % bold for greek characters
\newcommand{\bbR}{\mathbb{R}}

\author{Nobuyuki Umetani}

\begin{document}


\title{Boundary Element Method\\ 
Helmhotlz Equatoin}

\maketitle


\section{Boundary Formulation of Acoustics}
\label{sec:boundary_element_forumlation}
To make the paper self-contained, we briefly explain the boundary formulation of the Helmholtz equation.
%
We refer readers to the book~\cite{Kirkup199810} for more details of the BEM implementation.
%
The Helmholtz equation~(1)~\if0\eqref{eqn:helmholtz}\fi has a kernel
%
\begin{eqnarray}
\label{eqn:kernel}
G(\bi{x},\bi{y}) = \frac{\exp(+ikr)}{4\pi r},\;\;\;\;\mathrm{where}\;r=||\bi{x}-\bi{y}||,
\end{eqnarray}
%
which is the fundamental solution to the Dirac delta function $\delta(\bi{\bi{x}}-\bi{y})$.
%
Using this kernel function, the second Stoke's theorem leads to the equation which the sound pressure on the surface $p(\bi{x})$ needs to satisfy 
%
\begin{align}
\label{eqn:bem_surface}
\frac{\Omega(\bi{x})}{4\pi} p(\bi{x})  + \int_{S} \frac{\partial G(\bi{x},\bi{y})}{\partial \bi{n}(\bi{y})} p(\bi{y}) ds(\bi{y}) 
= G(\bi{x},\bi{x}_{src}),\;\;\bi{x}\in S,
\end{align}
%
where the $\partial G(\bi{x},\bi{y}) / \partial \bi{n}(\bi{y})$ derivative the kernel with respect to change of $\bi{y}\in S$ in the normal direction of the surface is
%
\begin{eqnarray}
\label{eqn:grad_kernel}
\frac{\partial G(\bi{x},\bi{y})}{\partial \bi{n}(\bi{y})}
=
\frac{\exp(+ikr)}{4\pi r^2}\left(1-ikr\right) \frac{(\bi{x}-\bi{y})\cdot \bi{n}}{||\bi{x}-\bi{y}||}.
\end{eqnarray}
%
The $\Omega(\bi{x})$ is a solid angle which takes $2\pi$ on a smooth surface, and is computed for triangle mesh using a formula presented in~\cite{van1983solid}.
%
In our implementation, the sound pressure is stored at the vertices of a triangle mesh and linearly interpolated over the triangle faces.
%
We discretize equation \eqref{eqn:bem_surface} using a typical collocation method, which formulates a linear system \if0\eqref{eqn:matrix_form}\fi(3)~by satisfying the equation at every vertex. 
%
We use a fifth-order Gaussian quadrature to compute this surface integration. 


Once the reflection pressure at the vertices $\bi{p}$ in (3)\if0\eqref{eqn:matrix_form}\fi~ is solved, the pressure value at the observation point $\bi{x}_{obs}$ inside medium $\Omega$ is computed with the surface integration 
%
\begin{align}
\label{eqn:bem_interior}
p(\bi{x}_{obs}) 
= 
- \int_S  \frac{\partial G(\bi{x}_{obs},\bi{y})}{\partial \bi{n}(\bi{y})} p(\bi{y}) ds(\bi{y})
 + G(\bi{x}_{obs},\bi{x}_{src}),\;\;\bi{x}_{obs}\in \Omega.
\end{align}


Our implementation is specifically categorized as the conventional boundary integration method (CBIM), in contrast to a more sophisticated model such as the Burton-Miller method~\cite{burton1971application}. 
%
The CBIM often suffers from errors in the frequency where the complementary region of the media $\bar{\Omega}=\{\bi{x}\in\mathbb{R}^3| \bi{x}\notin \Omega \}$ has a fictitious resonance mode. 
%
In our simulation the complementary region $\bar{\Omega}$ is the solid region of the musical instrument.
%
Since our complementary region $\bar{\Omega}$ is small compared to the cavity, the fictitious resonance mode is much higher compared to the fundamental cavity resonance frequency, and thus CBIM is adequate.





The off-diagonal $(i,j)$-entry of the resulting coefficient matrix $A_{ij}$ is approximately written as:
%
\begin{eqnarray}
\label{eqn:mat_entry}
A_{ij} 
\simeq
\left[ \frac{\bi{r}_{ij}\cdot \bi{n}_i}{4\pi r^3_{ij}}\Delta_j\right]  \underbrace{\exp(+ikr_{ij})\left(1-ikr_{ij}\right)}_{g(\gamma)},
\end{eqnarray}
%
where $\bi{r}_{ij}$ is a vector between $i$- and $j$-vertices, $r_{ij}=||\bi{r}_{ij}||$, the $\bi{n}_i$ is the unit normal vector, $\Omega_i$ is the solid angle at the $i$-vertex, and $\Delta_j$ is one third of the area of triangles around $j$-vertex. 
%
Notice the nonlinearity of the coefficient matrix with respect to wavenumber $k$ (see Sec.~5.1\if0\ref{subsec:back_ground}\fi). 
%
Furthermore, the nonlinear dependent part $g(\gamma)$ is a function of $\gamma=k r_{ij}$ and if it is small, the linear approximation over the wavenumber is reasonable (see Sec.~6.2\if0\ref{subsec:sampling}\fi).
%
Finally, the entry is invariant under the scaling geometry with $s$ and scaling the wave number with $1/s$ i.e., $r_{ij}\rightarrow s r_{ij}$ and  $k\rightarrow k/s $ (see Sec.~8\if0\ref{sec:detail}\fi).








\end{document}




