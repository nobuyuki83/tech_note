\documentclass[12pt]{article}
\usepackage{geometry}
\geometry{a4paper}


\usepackage{color}
\usepackage{hyperref}
\usepackage{amsmath}
\usepackage{amsfonts}
\usepackage{amssymb}
\usepackage{graphicx}
\usepackage{tcolorbox}
\usepackage{listings}
\usepackage{here}
\usepackage{txfonts}
\usepackage{algorithm}
\usepackage{algorithmic}
\usepackage{siunitx}
\usepackage{xcolor}

\lstset {language = c++,
  basicstyle = \ttfamily \scriptsize,
  commentstyle = \textit,
  frame = tRBl,
  framesep = 5pt,
  showstringspaces = false,
  numbers = left,
  stepnumber = 1,
  numberstyle = \tiny,
  tabsize = 2,
  keywordstyle = \bfseries \color{blue},
  stringstyle=\color{magenta},
  commentstyle=\color{red},
  morecomment=[l][\color{red}]{\#}
  showstringspaces=false, % don't mark spaces in strings
}
\newcommand{\bi}[1]{\mathbf{#1}}
\newcommand{\bs}[1]{\boldsymbol{#1}}  % bold for greek characters
\newcommand{\bbR}{\mathbb{R}}

\author{Nobuyuki Umetani}

\title{Continuum Mechanics:\\Governing Equation of Fluid}


\begin{document}
\maketitle

\tableofcontents


%%%%%%%%%%%%%%%%%%%%%%%%%%%%%%%%%%%%%%%%%%%%%%%%%% %%%%%%%%%%%%%%
\section{Constituent formula of fluid}
%%%%%%%%%%%%%%%%

From the principle Ⅰ (principle of local action) on the composition formula,
The stress at the point $\bi{X}$ corresponds to the region near $\bi{X}$
From the past of the collection of substance stores to the current time $t$
It is determined by the history of exercise.
%
In particular, if a substance is a simple material whose stress $\bi{T}$ is determined depending only on the history of the deformation gradient $\bi{F}$, the following expression holds.

\begin{equation}
{\bi{T}}=\mathcal{F}_{s=0}^{\infty}({\bi{F}}(t-s))
\end{equation}
%%%%%%%%%%%%%%%%

In the composition formula of a simple substance,

\begin{equation}
\mathcal{F}_{s=0}^{\infty}({\bi{F}}(t-s))
=\mathcal{F}_{s=0}^{\infty}({\bi{F}}(t-s) \cdot {\bi{P}})
\end{equation}
%%%%%%%%%%%%%%%%

If there is some second-order tensor $\bi{P}$ to do,
This material is said to be symmetric with respect to transformation $\bi{P}$. $\bi{P}$
Represents the transformation of the object prior to the deformation $\bi{F}$,
It is called symmetric transformation.
It seems like liquid crystal has anisotropy in fluid
Generally, unless the density of the fluid changes unless it is a special case
Because the mechanical properties should also be the same, $\bi{P}$

\begin{equation}
\det{\bi{P}}=1
\end{equation}
%%%%%%%%%%%%%%%%

When it is an arbitrary transformation that satisfies the condition, the substance is defined as a (simple) fluid
be able to. Conversion satisfying the above equation
It is called unimodular transformation.
Since the arbitrary rotation satisfies the above equation, it is understood that the fluid is isotropic.
In the above expression $\bi{P}$

\begin{equation}
{\bi{P}}=J\det{\bi{F}^{-1}}(t)
\end{equation}
%%%%%%%%%%%%%%%%

, The composition formula of the fluid is
\begin{equation}
{\bi{T}}=\mathcal{F}_{s=0}^{\infty}\big({\bi{F}}(t-s) \cdot J {\bi{F}^{-1}}(t)\big)
\end{equation}
%%%%%%%%%%%%%%%%

It is necessary that one is established. At the same time, if the above equation holds,
For any $\det{\bi{P}}\neq0$ conversion $\bi{P}$
Substituting $\tilde{\bi{F}}(t-s)={\bi{F}}(t-s)\cdot{\bi{P}}$ for the above equation

\begin{eqnarray}
\tilde{\bi{T}}
&=&\mathcal{F}_{s=0}^{\infty}\big(\tilde{\bi{F}}(t-s) \cdot J \tilde{\bi{F}^{-1}}(t)\big)\\  							&=&\mathcal{F}_{s=0}^{\infty}\big({\bi{F}}(t-s)\cdot{\bi{P}} \cdot J ({\bi{F}}(t)\cdot{\bi{P}})^{-1}\\  				&=&\mathcal{F}_{s=0}^{\infty}\big({\bi{F}}(t-s)\cdot{\bi{P}} \cdot J {\bi{P}^{-1}} \cdot {\bi{F}^{-1}}(t)\big)\\  		&=&\mathcal{F}_{s=0}^{\infty}\big({\bi{F}}(t-s) \cdot J {\bi{F}^{-1}}(t)\big)\\
&=&{\bi{T}}
\end{eqnarray}
%%%%%%%%%%%%%%%%

Since it can be expressed in a form independent of the conversion $\bi{P}$,
It is fully satisfied as a constituent formula of fluid.
From the above, it is necessary and sufficient condition to be able to write constituent formula of (simple) fluid as follows.

\begin{equation}
{\bi{T}}=\mathcal{F}_{s=0}^{\infty}\big({\bi{F}}(t-s) \cdot J {\bi{F}^{-1}}(t)\big)
\end{equation}
%%%%%%%%%%%%%%%%

Next, the inside of the parenthesis on the right side of the above expression
\begin{eqnarray}
{\bi{F}}(t-s) \cdot J {\bi{F}^{-1}}(t)
&=&  {\bi{F}}(t) \cdot J {\bi{F}^{-1}}(t)\\
	&&+ \frac{\partial{\bi{F}}(t)}{\partial t} \cdot J {\bi{F}^{-1}}(t) ds\\
	&&+ \frac{1}{2!}\frac{\partial^2{\bi{F}}(t)}{\partial^2t} \cdot J{\bi{F}^{-1}}(t)ds^2\\
	&&+ \cdots\\
	&=&  J + J {\bi{L}} ds + J \frac{1}{2!} {\bi{L}_{(2)}}  ds^2 + \cdots
\end{eqnarray}
%%%%%%%%%%%%%%%%

You can expand Taylor like.
However, $\bi{L}$ is the velocity gradient tensor, $\bi{L}_{(n)}$
Is the velocity gradient tensor on the n th floor. If up to 2 terms on the right side of the above equation
If the residual term becomes small enough, the composition formula can be written as follows.

\begin{equation}
{\bi{T}}  =  {\bi{f}}\big(J,J{\bi{L}}\big)\\  				=  {\bi{f}}\big(J,{\bi{L}}\big)
\end{equation}
%%%%%%%%%%%%%%%%

Further, the volume change rate from the reference arrangement, that is Jacobian and density,
Since Jacobian is associated from the above equation, instead of the density of the current time
It can be a function. In other words,

\begin{equation}
{\bi{T}}={\bi{f}}\big(\rho,{\bi{L}}\big)
\end{equation}
%%%%%%%%%%%%%%%%

Become
%%%%%%%%%%%%%%%%

Here, when the velocity of the fluid is uniform, that is, the velocity gradient is 0
Consider stress $\tilde{\bi{T}}$ of
When the velocity gradient is 0, it is $\bi{L}=0$, so from the above equation

\begin{equation}
\tilde{\bi{T}}={\bi{f}}(\rho)
\end{equation}

%%%%%%%%%%%%%%%%

. Also, if $\bi{Q}$ is an arbitrary symmetric tensor from the principle of material objectivity,

\begin{equation}
\tilde{\bi{T}}^*  =  {\bi{Q}} \tilde{\bi{T}} {\bi{Q}^{T}}  =  {\bi{f}}(\rho^*)  =  {\bi{f}}(\rho)  =  \tilde{\bi{T}}
\end{equation}
%%%%%%%%%%%%%%%%

For any rotation by symmetric tensor $\bi{Q}$
The invariant tensor is limited to a constant multiple of the unit tensor $\bi{I}$.
Therefore, the stress $\tilde{\bi{T}}$ in this case is

\begin{equation}
\tilde{\bi{T}}=-p(\rho){\bi{I}}
\end{equation}
%%%%%%%%%%%%%%%%

. The stress $\tilde{\bi{T}}$ when the velocity of the fluid is uniform
Hydrostatic stress, $p$ hydrostatic pressure,
. Also, as in the above formula, the pressure is determined only from the density
The fluid is called Barotoropic fluid or forward pressure fluid.
Also, when a fluid whose pressure depends on temperature or the like other than density
It is called baroclinic fluid. Also, as with incompressible fluids, pressure
In the case where it acts as a restraining force to satisfy incompressibility, this pressure is referred to as ambient hydrostatic pressure.

From the stress-determining function $\bi{f}$ this hydrostatic stress $\tilde{\bi{T}}$
If you define a new function without $\bi{f}$,

\begin{equation}
{\bi{T}}=-p(\rho){\bi{I}}+{\bi{f}}(\rho,{\bi{L}})
\end{equation}
%%%%%%%%%%%%%%%%

. However, $\bi{f}$ has a uniform fluid velocity
From the stress $\bi{T}$ the stress when the velocity gradient is 0
Since it is the excluded one, when the velocity gradient is 0, $\bi{f}=0$
Must be satisfied. In other words,

\begin{equation}
{\bi{f}}(\rho,0)=0
\end{equation}
%%%%%%%%%%%%%%%%

. And if $\bi{f}=0$ always holds,

\begin{equation}
{\bi{T}}=-p(\rho){\bi{I}}
\end{equation}
%%%%%%%%%%%%%%%%

%%%%%%%%%%%%%%%%
, This constituent formula represents an ideal fluid.
Although the basic physical properties of a fluid are that it can not hold shear stress when in parallel,
The fluid represented by such a constituent formula
It is ideal in the sense that it can preserve this property during exercise.
The absence of shear stress in the ideal fluid means that,
In simple shear motion, adjacent layers do not have resistance
It means that you can slide.
In a viscous fluid, this type of relative motion has frictional resistance
Since shear force is generated, the ideal fluid is inviscid fluid,
It can be safely said.

In the case of non-ideal fluid it is $\bi{f}\ne0$,
Since $\bi{f}$ represents the displacement stress from the ideal fluid
$\bi{f}$ is called overstress.
Generally, the overstress represents stress due to viscosity.

The symmetric component of the velocity gradient tensor $\bi{L}$ is $\bi{D}$,
Assuming that the antisymmetric component is $\bi{W}$,

\begin{eqnarray}
{\bi{L}} = {\bi{D}} + {\bi{W}}
\end{eqnarray}
%%%%%%%%%%%%%%%%

. $\bi{D}$ is the deformation rate tensor
$\bi{W}$ is called a spin tensor.
$\bi{W}$ represents rigid body rotation and is not related to stress That is,
It is considered that the composition formula can be represented only by the deformation rate tensor $\bi{D}$.
Actually, from the principle of material objectivity,
The function $\bi{f}$ is an arbitrary orthogonal tensor $\bi{Q}$
And against $\dot{\bi{Q}}$ which becomes an arbitrary antisymmetric tensor by its time differentiation

\begin{eqnarray}
{\bi{T}^*}
&=&{\bi{Q}}{\bi{T}}{\bi{Q}^T}\\
&=&  -p(\rho){\bi{I}}+{\bi{Q}}{\bi{f}}\big(\rho,{\bi{L}}\big){\bi{Q}^T}\\
&=&  -p(\rho){\bi{I}}+{\bi{f}}\big(\rho,{\bi{L}^*}\big)\\
&=&  -p(\rho){\bi{I}}+{\bi{f}}\big(\rho,{\bi{Q}}{\bi{L}}{\bi{Q}^T}+\dot{\bi{Q}}{\bi{Q}^T})
\end{eqnarray}
%%%%%%%%%%%%%%%%

Must be satisfied.
Since $\bi{Q}$ and $\dot{\bi{Q}}$ were arbitrary,
Here $\bi{Q}={\bi{I}}$, $\dot{\bi{Q}}$ to $\bi{L}$
Minus the antisymmetric component of
With $\dot{\bi{Q}}=-({\bi{L}})_a=-{\bi{W}}=-\frac{1}{2}({\bi{L}}-{\bi{L}^T})$

\begin{eqnarray}
{\bi{f}}\big(\rho,{\bi{L}}\big)={\bi{f}}\big(\rho,{\bi{L}}-({\bi{L}})_a\big)
={\bi{f}}\big(\rho,({\bi{L}})_s\big)={\bi{f}}\big(\rho,{\bi{D}}\big)
\end{eqnarray}
%%%%%%%%%%%%%%%%

Therefore,

\begin{eqnarray}
{\bi{T}}=-p({\rho}){\bi{I}}+{\bi{f}}\big(\rho,{\bi{D}}\big)
\end{eqnarray}
%%%%%%%%%%%%%%%%

, And the excessive stress $\bi{f}$ is the strain rate tensor $\bi{D}$
It becomes a function of

The antisymmetric component $\bi{W}$ of the velocity gradient tensor $\bi{L}$ is
Although it came out, in general the antisymmetric tensor $\bi{B}$
An axial vector

\begin{eqnarray}
{\bs{\omega}}=-\frac{1}{2}{\epsilon_{ijk}}{B_{ij}}{\bi{e}_k}
\end{eqnarray}
%%%%%%%%%%%%%%%%

For any vector $\bi{a}$
$\bi{B}\cdot{\bi{a}}=\omega\times{\bi{a}}$
. Against the spin tensor $\bi{W}$
The axial vector $\bi{\omega}$

\begin{eqnarray}
{\bi{\omega}}
=  -\frac{1}{2}{\epsilon_{ijk}}{W_{jk}}{\bi{e}_i}
=  \frac{1}{2}{\epsilon_{ijk}}{\frac{\partial\bi{v}_k}{\partial\bi{x}_j}}{\bi{e}_i}
=  \frac{1}{2}{\nabla_x \times {\bi{v}}}
\end{eqnarray}

, And represents an angular velocity vector in which the fluid rotates rigidly.
From the above equation, the angular velocity vector $\bi{\omega}$ is
Defined in $\bi{\Omega} = \nabla_x \times {\bi{v}}$
It is half the vorticity vector.


%%%%%%%%%%%%%%%%%%%%%%%%%%%%%%%%%%%%%%%%%%%%%%%%%% %%%%%%%%%%%%%%
\section{incompressible expression}
%%%%%%%%%%%%%%%%
Assuming incompressible conditions for the fluid, the mass density of the material point $\bi{X}$ is
Since it is constant regardless of time, the following condition is satisfied.

\begin{eqnarray}
\left.\frac{\partial\rho}{\partial t}\right|_{\bi{X}} = 0
\end{eqnarray}

Therefore, if this is adapted to a continuous equation, for Lagrange display, Eular display
The incompressible continuity formula is as follows.

\begin{eqnarray}
\nabla_x \cdot {\bi{v}} = 0
\end{eqnarray}

\begin{eqnarray}
\left.\frac{\partial\rho}{\partial t}\right|_{\bi{x}}+{\bi{v}}\cdot(\nabla_x\rho)=0
\end{eqnarray}

Furthermore, in the case of a single fluid, or when the density of the fluid is spatially constant

\begin{eqnarray}
\rho = Const
\end{eqnarray}

, The incompressibility condition is always satisfied,
Continuous expressions have meaning only in the Lagrange display format.
That is, the Eular indication of continuous equations is always satisfied.
From the above, the incompressibility equation of constant density fluid is as follows.

\begin{eqnarray}
\nabla_x \cdot {\bi{v}} = 0
\end{eqnarray}

\begin{tcolorbox}[title=incompressible expression]
\begin{eqnarray}
\nabla_x \cdot {\bi{v}} = 0
\end{eqnarray}
\end{tcolorbox}


%%%%%%%%%%%%%%%%%%%%%%%%%%%%%%%%%%%%%%%%%%%%%%%%%% %%%%%%%%%%%%%%
\section{Fluid pressure}

Constraints may exist in deformation of a substance.
Such restrictions represented by incompressibility conditions etc.
It is called internal constraint.
If there is an internal bond in the object
The principle Ⅰ (principle of stress determination) on composition formula needs to be transformed as follows.

\begin{eqnarray}
{\bi{T}}=\mathcal{F}_{s=0}^{\infty}({\bi{F}}(t-s))+\mathcal{T}
\end{eqnarray}
%%%%%%%%%%%%%%%%

This stress $\mathcal{T}$ is irrelevant to internal work,
From the movement of material points, that is, the history of $\bi{F}$
It is a stress that can not be decided
This is called interminate stress.
Since nondeterministic stress does not work

\begin{eqnarray}
\mathcal{T} : {\bi{D}} = \mathrm{tr}( \mathcal{T} \cdot {\bi{D}}) = 0
\end{eqnarray}
%%%%%%%%%%%%%%%%

Is satisfied. A typical internal constraint is a scalar value tensor function

\begin{eqnarray}
\gamma({\bi{F}}) = 0
\end{eqnarray}
%%%%%%%%%%%%%%%%

As shown in FIG. For example, as a representative example thereof, an uncompressed constraint condition

\begin{eqnarray}
\gamma({\bi{F}}) = \det{\bi{F}} - 1 = J-1 = 0
\end{eqnarray}
%%%%%%%%%%%%%%%%

.

Since internal work must also satisfy the principle of substance objectivity,

\begin{eqnarray}
\gamma^*({\bi{F}}) = \gamma({\bi{F}}) = \gamma({\bi{Q}} \cdot {\bi{F}})
\end{eqnarray}

%%%%%%%%%%%%%%%

Must be satisfied.
Here, if you say $\bi{Q}={\bi{R}^T}$

\begin{eqnarray}
\gamma({\bi{F}}) = \gamma({\bi{R}^T} \cdot {\bi{R}} \cdot {\bi{U}})  =  \gamma({\bi{U}})
\end{eqnarray}
%%%%%%%%%%%%%%%%

$\gamma$ is the right stretch tensor $\bi{U}$ only
It is necessary to be a function.
While the right stretch tensor is invariant to the rotation of the reference frame
In other words, because it is $\bi{U}^*=\bi{U}$

\begin{eqnarray}
\gamma^*({\bi{U}})=\gamma({\bi{U}}^*)
\end{eqnarray}
%%%%%%%%%%%%%%%%

It seems that it satisfies objectivity sufficiently.
Therefore, constraint condition $\gamma$ is only right stretch tensor $\bi{U}$
It must be a function of.

Here, what non-deterministic stress is imposed on incompressible restraint conditions
It is examined whether it occurs. $\gamma$ was 0 and was a constant
Time-differentiated $\lambda$ should also be zero. Accordingly

\begin{eqnarray}
\dot{\gamma}
=  (\det{\bi{F}}-1)^{\dot{}}
=  \dot{J}
=  J\mathrm{tr}{\bi{D}}
=  \mathrm{tr}{\bi{D}}
=  0
\end{eqnarray}
%%%%%%%%%%%%%%%%

Here, we used $J=1$ from the constraint condition. From the above equation

\begin{eqnarray}
\mathrm{tr}{\bi{D}}={\bi{I}}:{\bi{D}}=\mathrm{tr}({\bi{I}}\cdot{\bi{D}})=0
\end{eqnarray}
%%%%%%%%%%%%%%%%

. Therefore, when incompressible constraints are imposed from the above equation
The stress represented by a constant multiple of the unit tensor is
I understand that I do not work.
Thus, the unknown stress $\mathcal{T}$ is calculated using the unknown number $p$

\begin{eqnarray}
\mathcal{T}=-p{\bi{I}}
\end{eqnarray}
%%%%%%%%%%%%%%%%

Can be expressed as follows. $p$ introduced here
It is clearly indeterminate hydrostatic pressure.


% Fluids whose pressure is determined by density are called Barotoropic fluid.
% In the case of analysis of fine compression in which fluid is hardly compressed,


%%%%%%%%%%%%%%%%%%%%%%%%%%%%%%%%%%%%%%%%%%%%%%%%%% %%%%%%%%%%%%%%
\section{Constituent formula of Newton fluid}

A fluid whose overstress does not depend on density
Storks fluid (Storkesian fluid), its constituent formula is

\begin{eqnarray}
{\bi{T}} = -p{\bi{I}} +{\bi{f}}({\bi{D}})
\end{eqnarray}
%%%%%%%%%%%%%%%%

.
$p$ is the pressure, $\bi{I}$ is the unit tensor.
Since the tensor $\bi{D}$ is an objective tensor,
According to the principle of material objectivity the function $\bi{f}$

\begin{eqnarray}
 {\bi{Q}} \cdot {\bi{f}}({\bi{D}}) \cdot {\bi{Q}^T} = {\bi{f}}( {\bi{Q}} \cdot {\bi{D}} \cdot {\bi{Q}^T} )
\end{eqnarray}

Must be satisfied.
That is, since $\bi{f}$ is an isotropic tensor function, according to the display theorem,
$\phi_i$ as the scalar function of the main invariant of $\bi{D}$

\begin{eqnarray}
{\bi{f}}({\bi{D}}) = \phi_0{\bi{I}} + \phi_1 {\bi{D}} + \phi_2 {\bi{D}}^2
\end{eqnarray}

It can be expressed as. The fluid with overstressing the above equation is Reiner-Rivlin fluid
Also called. The third term of the above equation is a case where the external force condition suddenly changes
It is known to have an impact,
Even in fluids that generally exhibit strong nonlinearity
It is known that it is almost negligibly small.

Overstress $\bi{f}(\bi{D})$ is $\bi{D}$
A fluid that is linearly homogeneous with respect to is called a Newton fluid.
In the above equation
$\phi_1=\lambda(tr{\bi{D}})$,
$\phi_2=2\mu$,
If you say $\phi_3=0$

\begin{eqnarray}
{\bi{f}}({\bi{D}})=\lambda(tr{\bi{D}}){\bi{I}} + 2\mu{\bi{D}}
\end{eqnarray}
%%%%%%%%%%%%%%%%

, The composition formula of the Newton fluid can be expressed as follows.

\begin{eqnarray}
{\bi{T}}
&=&  -p {\bi{I}} + \lambda(tr{\bi{D}}){\bi{I}} + 2\mu{\bi{D}}\\
&=& -p {\bi{I}} + \lambda(\nabla_x \cdot {\bi{v}}){\bi{I}} + 2\mu{\bi{D}}
\end{eqnarray}
%%%%%%%%%%%%%%%%

Here, $\mu$ is the shear viscosity,
$\lambda$ is called the second viscosity.
The second viscosity index $\lambda$ is like the first constant of Lame in a linear elastic body.

% For the second viscosity rate, the following link is detailed
%: International College of FEM: FEM: Fluid Dynamics: Second Coefficient of Viscosity | http: //www.fem.gr.jp/fem/fluid/2ndviscosity/2ndviscosity.html

\begin{tcolorbox}[title=Newton structural formula of the fluid]
\begin{eqnarray}
\bi{T} = -p {\bi{I}} + \lambda(\nabla_x \cdot {\bi{v}}){\bi{I}} + 2\mu{\bi{D}}
\end{eqnarray}
\end{tcolorbox}


The above equation is the deviation component of $\bi{D}$ $\tilde{\bi{D}}$

\begin{eqnarray}
\tilde{\bi{D}}
=  {\bi{D}} - \frac{1}{3}(\mathrm{tr}{\bi{D}}){\bi{I}}\\
=  {\bi{D}} - \frac{1}{3}(\nabla_x \cdot {\bi{v}}){\bi{I}}
\end{eqnarray}
%%%%%%%%%%%%%%%%

Looking at

\begin{eqnarray}
{\bi{T}}  = -p{\bi{I}} + (\lambda+\frac{2}{3}\mu)(tr{\bi{D}}){\bi{I}}+2\mu\tilde{\bi{D}}
\end{eqnarray}
%%%%%%%%%%%%%%%%

. Here's $\lambda+\frac{2}{3}\mu$
It is called bulk viscosity $\kappa$.
Bulk viscosity is a measure of viscous stress accompanying volume change of fluid.
The above expression uses $\kappa$

\begin{eqnarray}
{\bi{T}} = -p{\bi{I}} + \kappa(\nabla_x \cdot {\bi{v}}){\bi{I}}+2\mu\tilde{\bi{D}}
\end{eqnarray}
%%%%%%%%%%%%%%%%

It can be shown.
For many fluids $\kappa$ is known to be very small.
So, using the approximation of Stokes to be $\kappa=0$
The constituent formula of the compressible fluid is as follows

\begin{eqnarray}
{\bi{T}}
&=&  -p{\bi{I}} + 2\mu\tilde{\bi{D}}\\
&=&  -p{\bi{I}}+2\mu( {\bi{D}} - \frac{1}{3}(\nabla_x \cdot {\bi{v}}){\bi{I}} )
\end{eqnarray}
%%%%%%%%%%%%%%%%

This is a constitutive expression of the compressible Newton fluid ignoring the volume viscosity coefficient.
Furthermore, assuming incompressibility in the fluid

\begin{eqnarray}
\nabla_x \cdot {\bi{v}} = 0
\end{eqnarray}
%%%%%%%%%%%%%%%%

So, the constituent formula of incompressible Newton fluid is

\begin{eqnarray}
{\bi{T}} = -p{\bi{I}} + 2\mu{\bi{D}}
\end{eqnarray}
%%%%%%%%%%%%%%%%

.

\begin{tcolorbox}[title=constituent formula of incompressible Newton fluid]
\begin{eqnarray}
{\bi{T}} = -p{\bi{I}} + 2\mu{\bi{D}}
\end{eqnarray}
\end{tcolorbox}


%%%%%%%%%%%%%%%%%%%%%%%%%%%%%%%%%%%%%%%%%%%%%%%%%% %%%%%%%%%%%%%%
\section{Navier-Storks equation}

Constitution formula of Newton fluid is as follows

\begin{eqnarray}
{\bi{T}}=-p{\bi{I}}+\lambda(tr{\bi{D}}){\bi{I}}+ 2 \mu {\bi{D}}
\end{eqnarray}
%%%%%%%%%%%%%%%%

$\mu$ is the viscosity coefficient. This is described by Cauchy's first law of motion described in the Eulerian coordinate system

\begin{eqnarray}
\rho\left. \frac{\partial \bi{v}}{\partial t} \right|_{\bi{x}}+\rho({\bi{v}} \otimes \nabla ) \cdot {\bi{v}}=\rho {\bi{g}}+ \nabla_x \cdot {\bi{T}}
\end{eqnarray}
%%%%%%%%%%%%%%%%

Substitute for

\begin{eqnarray}
\rho\left. \frac{\partial \bi{v}}{\partial t} \right|_{\bi{x}}+\rho({\bi{v}} \otimes \nabla_x ) \cdot {\bi{v}}=\rho {\bi{g}}+ \nabla_x \cdot (-p{\bi{I}}+\lambda(tr{\bi{D}}){\bi{I}}+2\mu{\bi{D}})
\end{eqnarray}
%%%%%%%%%%%%%%%%

here

\begin{eqnarray}
\nabla_x \cdot (-p{\bi{I}})  =  -\frac{\partial}{\partial x_k}{\bi{e}}_k\cdot( p \delta_{ji}{\bi{e}}_j \otimes {\bi{e}}_i )  =  -\frac{\partial p}{\partial x_k}({\bi{e}}_k \cdot {\bi{e}}_j )\delta_{ji}{\bi{e}}_i \\			=  -\frac{\partial p}{\partial x_k}\delta_{kj}\delta_{ij}{\bi{e}}_i  =  -\frac{\partial p}{\partial x_i}{\bi{e}}_i						=  -\nabla_x p
\end{eqnarray}
%%%%%%%%%%%%%%%%

Substituting

\begin{eqnarray}
\rho\left.\frac{\partial {\bi{v}}}{\partial t} \right|_x + \rho({\bi{v}} \otimes \nabla_x ) \cdot {\bi{v}}  =  \rho {\bi{g}}+-\nabla_x p+\nabla \cdot (\lambda(tr{\bi{D}}){\bi{I}}+2\mu{\bi{D}})
\end{eqnarray}
%%%%%%%%%%%%%%%%

We call this a stress emanating form of the Navier-Storks equation. Here, we deform the stress-divergent Navier-Storks equation for the case where the viscosity coefficient $\mu$ and the second viscosity coefficient $\lambda$ are constant in the fluid.

\begin{eqnarray}
\nabla_x \cdot ( \nabla \otimes {\bi{v}} )
&=&  \frac{\partial}{\partial x_k}{\bi{e}}_k \cdot ( \frac{\partial v_i}{\partial x_l} {\bi{e}}_l \otimes {\bi{e}}_i ) \\
&=&  \frac{\partial}{\partial x_k}\frac{\partial v_i}{\partial x_l}( {\bi{e}}_l \cdot {\bi{e}}_i ){\bi{e}}_i \\
&=&  \frac{\partial}{\partial x_k}\frac{\partial v_i}{\partial x_l}( \delta_{kl} ){\bi{e}}_i  \\
&=&  \frac{\partial}{\partial x_k}\frac{\partial v_i}{\partial x_k}{\bi{e}}_i \\
&=&  \nabla_x^2 {\bi{v}}
\end{eqnarray}
%%%%%%%%%%%%%%%%

\begin{eqnarray}
\nabla \cdot ( {\bi{v}} \otimes \nabla )
&=&  \frac{\partial}{\partial x_k}{\bi{e}}_k \cdot ( \frac{\partial v_l}{\partial x_i} {\bi{e}}_l \otimes {\bi{e}}_i )\\
&=&  \frac{\partial}{\partial x_k}\frac{\partial v_k}{\partial x_i}{\bi{e}}_i  \\
&=&  \frac{\partial}{\partial x_i} \cdot \frac{\partial v_k}{\partial x_k}{\bi{e}}_i \\
&=&  \frac{\partial}{\partial x_i}{\bi{e}}_i( \frac{\partial}{\partial x_k}{\bi{e}}_k \cdot {\bi{v}}_l ) \\
&=&  \nabla( \nabla \cdot {\bi{v}} )
\end{eqnarray}
%%%%%%%%%%%%%%%%

Therefore, divergence of $\bi{D}$ is transformed as follows.

\begin{eqnarray}
\nabla_x \cdot {\bi{D}}
&=&  \nabla_x \cdot\frac{1}{2}({\bi{v}} \otimes \nabla_x+\nabla_x \otimes {\bi{v}}) \\
&=&  \frac{1}{2}( \nabla^2 {\bi{v}}+\nabla ( \nabla_x \cdot {\bi{v}} ) )
\end{eqnarray}
%%%%%%%%%%%%%%%%

further,

\begin{eqnarray}
\nabla_x \cdot (tr{\bi{D}}){\bi{I}}  =  \nabla_x ( \nabla_x \cdot {\bi{v}} )
\end{eqnarray}
%%%%%%%%%%%%%%%%

The viscosity coefficient and the second viscosity coefficient were made constant in the fluid. Substituting into the Navier-Stokes equations of stress divergence form, the Navier-Stokes equation is

\begin{eqnarray}
\rho\left.\frac{\partial {\bi{v}}}{\partial t} \right|_x + \rho({\bi{v}} \otimes \nabla_x ) \cdot {\bi{v}}  =  \rho {\bi{g}}+ -\nabla_x p+\mu \nabla^2 {\bi{v}}+(\mu+\lambda) \nabla_x ( \nabla_x \cdot {\bi{v}} )
\end{eqnarray}
%%%%%%%%%%%%%%%%

Substitute $\kappa=\lambda+\frac{2}{3}\mu$

\begin{eqnarray}
\rho\left.\frac{\partial {\bi{v}}}{\partial t} \right|_x + \rho({\bi{v}} \otimes \nabla_x ) \cdot {\bi{v}}  =  \rho {\bi{g}}+ -\nabla_x p+\mu \nabla^2 {\bi{v}}+(\kappa+\frac{1}{3}\mu) \nabla_x ( \nabla_x \cdot {\bi{v}} )
\end{eqnarray}
%%%%%%%%%%%%%%%%

. Here, when Stokes's assumption $\kappa=0$ holds,
The Navier-Stokes equation is multiplied as follows.

\begin{eqnarray}
\rho\left.\frac{\partial {\bi{v}}}{\partial t} \right|_x + \rho({\bi{v}} \otimes \nabla_x ) \cdot {\bi{v}}  =  \rho {\bi{g}} + -\nabla_x p + \mu \nabla^2 {\bi{v}}+\frac{1}{3}\mu \nabla_x ( \nabla_x \cdot {\bi{v}} )
\end{eqnarray}
%%%%%%%%%%%%%%%%

Further, in the case where incompressibility is established in the fluid
Since $\nabla_x \cdot {\bf{v}}=0$ holds

\begin{eqnarray}
\rho\left.\frac{\partial {\bi{v}}}{\partial t} \right|_x + \rho({\bi{v}} \otimes \nabla_x ) \cdot {\bi{v}}  =  \rho {\bi{g}}+ -\nabla_x p+\mu \nabla^2 {\bi{v}}
\end{eqnarray}

This is called Laplacian type incompressible Navier-Storks equation.


\begin{tcolorbox}[title=incompressible Navier-Storkes equation]
\begin{eqnarray}
\rho\left.\frac{\partial {\bi{v}}}{\partial t} \right|_x + \rho({\bi{v}} \otimes \nabla_x ) \cdot {\bi{v}}  =  \rho {\bi{g}}+ -\nabla_x p+\mu \nabla^2 {\bi{v}}
\end{eqnarray}
\end{tcolorbox}

\if0
* References
** Books
| Fundamentals of Tensor Analysis for Nonlinear Finite Element Method | Hisada Toshiaki |
| "Continuum dynamics - a simple theory and example" | P. By Chadwick |
\fi


\end{document}



